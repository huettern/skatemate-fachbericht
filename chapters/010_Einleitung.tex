\chapter{Einleitung}
Die elektrische Mobilität ist auf dem Vormarsch. Sie ermöglicht eine geräusch- und emissionsarme Fortbewegung, die technischen Möglichkeiten werden immer vielfältiger. So beginnt die Elektromobilität auch den Freizeitbereich zu erobern. Es entsteht ein Schnittpunkt mit „der letzten Meile“, so wird eine kürzere Strecke auf dem Arbeitsweg eines Pendlers genannt, die nicht mit dem Bus befahren wird, zum Gehen eher zu lange ist, und um mit dem Fahrrad befahren zu werden müsste der Pendler ein zusätzliches Fahrrad an dieser Stelle haben – betrifft die letzte Meile jedoch einen Freizeitausflug, entfällt diese Möglichkeit. Wie können kürzere Strecken sonst noch überbrückt werden? Genau, mit einem elekrtrischen Skateboard! Dies kann problemlos überallhin mitgenommen werden, kürzere Strecken können schnell und elegant bewältigt werden. Das elektrische Skateboard ist also eine mögliche Lösung für das Problem der letzten Meile. \\
Im Rahmen dieses Projektes soll deshalb ein elektrisches Rollbrett entwickelt werden. Dabei ist ein innovatives Steuerungskonzept als auch eine effiziente Ansteuerung des Motors von Bedeutung, Diese beiden Punkte ermöglichen einen Beitrag an die Weiterentwicklung der Elektromobilität. Das Rollbrett als Endprodukt hat zum Ziel, im Freizeitbereich attraktiv zu sein. 
Weitere Anforderungen sind, dass der Antrieb ausgeschaltet werden kann, zudem muss dann das Board normal benutzt werden können. Zudem muss der Antrieb aus sicherheitstechnischen Gründen ausschalten, wenn niemand auf dem Board steht. Das Rollbrett muss wetterfest sein, so auch die Steuerung selbst. Diese kann vom Akku des Rollbretts oder von einer separaten Batterie gespeist sein.  \\
Die Lösung unseres Teams Skatemate ist das Longboard Commute. Das Konzept dazu wurde speziell für Pendler entwickelt. Die Steuerung erfolgt über die Bewegung des Zeigefingers (Geschwindigkeit) und über Gewichtsverlagerung(Richtung). Je nach Krümmung des Fingers beschleunigt der Motor oder bremst. Die technische Umsetzung erfolgt über einen Flex-Sensor. Die Ansteuerung des Motors erfolgt über die Feldorientierte Regelung (field oriented control FOC), dies ermöglicht ein sanftes Anfahren und gibt einen interessanten Einblick in eine Regelungstechnik, die auch bei industriellen Motoren angewendet wird. Damit der Akku nicht für jedes Laden vom Longboard gelöst werden muss, wird ein eigener, direkt am Deck integrierter Akkulader entwickelt, so dass das Longboard direkt über ein Ladekabel mit einer Steckdose verbunden werden kann. Dazu wird ein Balancing-System entwickelt.
\\ \\
Dieser Bericht stellt die technische Dokumentation des elektrischen Longboards Commute dar. Das Kapitel Überblick zeigt das Grobkonzept des Projektes auf. Zudem wird das Bedienkonzept vorgestellt. Das Kapitel Technische Grundlagen erklärt die benötigten Grundlagen für das Verständnis des Projektes. Dies beinhaltet die Funktionsweise des Flex-Sensors, des Balancing, der Feldorientierten Regelung, der H-Brücke und der Funkübertragung. In den Kapiteln Hardware und Software wird die Ausführung unsere Lösung im Detail präsentiert. Das Kapitel Validierung gibt Auskunft darüber, wie das System getestet wurde. 
