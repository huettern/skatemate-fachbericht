\chapter{Einleitung}
Die elektrische Mobilität ist auf dem Vormarsch. Sie ermöglicht eine geräusch- und emissionsarme Fortbewegung. Daber werden die technischen Möglichkeiten je länger je vielfältiger, sodass immer mehr und preiswertere elektrische Fortbewegungsmittel angeboten werden.
Die Elektromobilität erreicht auch immer mehr den Freizeitbereich.
Zur Weiterentwicklung der Elektromobilität soll im Rahmen dieses Projektes ein elektrisches Rollbrett entwickelt werden, das im Freizeitbereich attraktiv ist.
Gefordert ist ein innovatives Steuerungskonzept als auch eine effiziente Ansteuerung des Motors. Das Rollbrett als Endprodukt hat zum Ziel, im Freizeitbereich attraktiv zu sein und einen Beitrag an die Entwicklung der Elektromobilität zu leisten. Weitere Anforderungen sind, dass der Antrieb ausgeschaltet werden kann. Wenn niemand auf dem Rollbrett steht, muss er dies aus sicherheitstechnischen Gründen sogar. Zudem soll das Rollbrett normal benutzt werden können. 

Das Rollbrett muss wetterfest sein, so auch die Steuerung selbst. 

Diese kann vom Akku des Rollbretts oder von einer separaten Batterie gespeist sein.  \\
Die Lösung des Entwicklerteams Skatemate ist das Longboard Commute. Das Konzept dazu wurde speziell für Pendler entwickelt. Obwohl die Motorisierung immer mehr Bereiche durchdringt, sind die Möglichkeiten für Pendler stark eingeschränkt. Neben Elektrofahrräder und Segways, welche nur sehr mühsam oder mit Zusatzkosten mit dem Zug transportiert werden können, gibt es für Pendler kaum praktische Alternativen. Diese Lücke soll unser Produkt schliessen. Das Longboard ist mit einer innovativen Steuerung ausgestattet und mit einem schlanken Design versehen. Dadurch kann es problemlos überallhin mitgenommen werden, um kürzere Strecken schnell und elegant zu bewältigen. Somit wird dem Pendler ermöglicht, den letzten Abschnitt vom Bahnhof oder der Busstation zum Ziel elektrisiert zurückzulegen. \\
Die Steuerung erfolgt über die Bewegung des Zeigefingers (Geschwindigkeit) und über Gewichtsverlagerung(Richtung). Je nach Krümmung des Fingers beschleunigt der Motor oder bremst. Die technische Umsetzung erfolgt über einen Flex Sensor. Die Ansteuerung des Motors erfolgt über die Feldorientierte Regelung (field oriented control FOC), dies ermöglicht ein sanftes Anfahren und gibt einen interessanten Einblick in eine Regelungstechnik, die auch bei industriellen Motoren angewendet wird. Damit der Akku nicht für jedes Laden vom Longboard gelöst werden muss, wird ein eigener, direkt am Deck integrierter Akkulader entwickelt, so dass das Longboard direkt über ein Ladekabel mit einer Steckdose verbunden werden kann. Dazu wird ein Balancing-System entwickelt.
\\\\
Dieser Bericht stellt die technische Dokumentation des elektrischen Longboards Commute dar. Das Kapitel Überblick zeigt das Grobkonzept des Projektes auf. Zudem wird das Bedienkonzept vorgestellt. Das Kapitel Technische Grundlagen erklärt die benötigten Grundlagen für das Verständnis des Projektes. Dies beinhaltet die Funktionsweise des Flex-Sensors, des Balancing, der Feldorientierten Regelung, der H-Brücke und der Funkübertragung. In den Kapiteln Hardware und Software wird die Ausführung unsere Lösung im Detail präsentiert. Das Kapitel Validierung gibt Auskunft darüber, wie das System getestet wurde. 
