\chapter{Einleitung}
Die elektrische Mobilität ist auf dem Vormarsch. Sie ermöglicht eine geräusch- und emissionsarme Fortbewegung. Dabei werden die technischen Möglichkeiten je länger je vielfältiger, sodass immer mehr und preiswertere elektrische Fortbewegungsmittel angeboten werden.
Die Elektromobilität erreicht auch immer mehr den Freizeitbereich.
Zur Weiterentwicklung der Elektromobilität soll im Rahmen dieses Projektes ein elektrisches Rollbrett entwickelt werden, das im Freizeitbereich attraktiv ist.
Gefordert ist ein innovatives Steuerungskonzept als auch eine effiziente Ansteuerung des Motors. Das Rollbrett als Endprodukt hat zum Ziel, im Freizeitbereich attraktiv zu sein und einen Beitrag an die Entwicklung der Elektromobilität zu leisten. Weitere Anforderungen sind, dass der Antrieb ausgeschaltet werden kann. Wenn niemand auf dem Rollbrett steht, muss dies aus sicherheitstechnischen Gründen sogar automatisch geschehen. Zudem soll das Rollbrett auch ohne elektrischen Antrieb benutzt werden können. Für die Alltagstauglichkeit muss das Rollbrett wetterfest sein, so auch die Steuerung selbst.\\

Die Lösung des Entwicklerteams Skatemate ist das Longboard Commute. Das Konzept dazu wurde speziell für Pendler entwickelt. 
Obwohl die Motorisierung immer mehr Bereiche durchdringt, sind die Möglichkeiten für Pendler bisher stark eingeschränkt. Diese Lücke soll das Produkt Commute schliessen.
Als Kriterien ergibt sich dadurch, dass das Longboard gut tragbar, mobil zu laden und einfach zu steuern ist. Insbesondere soll die Steuerung keine zusätzlichen Fahrkenntnisse benötigen, also auch für Laien tauglich sein.\\
Deshalb ist das Longboard mit einer praktischen Steuerung ausgestattet und mit einem schlanken Design versehen. Dadurch kann es problemlos überallhin mitgenommen werden, um kürzere Strecken schnell und elegant zu bewältigen. Somit wird dem Pendler ermöglicht, den letzten Abschnitt von dem Bahnhof oder der Busstation zum Ziel elektrisiert zurückzulegen. 
\\ \\
Die Steuerung erfolgt über die Bewegung des Zeigefingers (Geschwindigkeit) und über Gewichtsverlagerung (Richtung). Je nach Krümmung des Fingers beschleunigt oder bremst der Motor. Die technische Umsetzung erfolgt über einen Flex-Sensor. \\
Der Motors wird mithilfe der feldorientierten Regelung (field oriented control FOC) angesteuert. Dadurch kann ein sanftes Anfahren realisiert werden. Zudem gibt die FOC einen interessanten Einblick in eine Regelungstechnik, die auch bei industriellen Motoren angewendet wird. \\
Damit der Akku nicht für jedes Laden von dem Longboard gelöst werden muss, wird ein eigener, direkt am Deck integrierter Akkulader entwickelt, so dass das Longboard direkt über ein Ladekabel mit einer Steckdose verbunden werden kann. Dazu wird ein Balancing-System entwickelt.
\\\\
Dieser Bericht stellt die technische Dokumentation des elektrischen Longboards Commute dar. Das Kapitel Grobkonzept und technische Grundlagen zeigt das Grobkonzept des Projektes auf und erklärt die benötigten technischen Grundlagen für das Verständnis des Projektes. Dies beinhaltet die Funktionsweise des Flex-Sensors, des Balancing, der feldorientierten Regelung, der H-Brücke und der Funkübertragung. In den Kapiteln Hardware und Software wird die Ausführung im Detail präsentiert. Das Kapitel Validierung gibt Auskunft darüber, wie das System getestet wurde. 
