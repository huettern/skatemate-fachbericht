\chapter{Schlusswort}

Während dieses intensiven, lehrreichen Projektes wurde ein elektronisches Skateboard entwickelt. Dabei wurde insbesondere eine innovative Steuerung erwartet. Diese Erwartung wurde erfüllt dank des Magic Gloves. Damit lässt sich die Geschwindigkeit des Skateboardes mit nur einem Finger durch Beugung und Streckung regulieren. Die Daten des dazu benötigten Flex-Sensors werden über Radiofunk zur Motoransteuerung geschickt. Die Motoransteuerung setzt diese Wunschbeschleunigungen um dank einer sensorlosen Feldorientierten Regelung( FOC). Diese Ansteuerung ermöglicht ein besonders sanftes Anfahren und dynamische Reaktionen auf Laständerungen. Versorgt wird das System dank eines LiPo-Akkus. Dieser soll zum Laden nicht vom Skateboard gelöst werden müssen, deshalb haben wir einen eigenen, im Skateboard integrierten Akkulader umgesetzt. Der Magic Glove verfügt über eine eigene Knopfbatterie.\\
\\
Die Umsetzung des \textbf{Magic Gloves} gelang ziemlich zufriedenstellend. Der Magic Glove kommuniziert kabellos mit der Motoranteuerung über das 2.4GHz-Funknetz. Die Messung der gewünschten Beschleunigung erfolgt mithilfe des Flex Sensors. Diese beiden Bereiche funktionieren. Die Genauigkeit der Messung müsste noch in Feldversuchen, welche aufgrund der nicht funktionierenden Motoransteuerung nicht durchgeführt werden konnten, optimiert werden. Das Steuerungskonzept wurde als innovative Idee bestätigt. \\
Die \textbf{Ansteuerung des Motors} bereitete einige Schwierigkeiten. Die sensorlose FOC konnte zwar softwaretechnisch prinzipiell gut umgesetzt werden, bei der effektiven Ansteuerung des Motors tauchten jedoch unvorhergesehene Probleme auf. Es stellte sich heraus, dass die Ansteuerung schwieriger als gedacht ist. Ein Problem sind dabei die grösstenteils unbekannten Motorparameter, die für die Einstellung des PID-Reglers benötigt werden. Wir konnten leider keine zufriedenstellende Lösung finden. Läuft der Motor zwangskommutiert, funktioniert die Ansteuerung. Geregelt, d.h. im closed loop, gibt es noch zu viele Probleme, was bedeutet, dass die Regelstruktur nicht zufriedenstellend umgesetzt ist. Es konnte sichergestellt werden, dass der Observer der Software läuft. Auch die Hardware konnte erfolgreich getestet und bestätigt werden. \\
Die Testversuche haben gezeigt, dass das Konzept selbst funktionieren würde, im Vergleich zu anderen Umsetzungen kann der Motor mit der FOC-Regelung auch bei tiefer Drehzahl gut betrieben werden. \\
Das \textbf{selbstkonzipierte Akkuladegerät} funktioniert neben der grundlegenden Laderegelung (erst laden mit konstantem Strom, anschliessend mit konstanter Spannung) über ein Balancing-System. Dies sorgt dafür, dass die einzelnen Zellen gleichmässig geladen werden. Der Prototyp funktioniert leider noch nicht wie gewünscht. Die Hardware funktioniert grösstenteils, das Balancing und das Laden konnte in Testläufen bestätigt werden. Die Software funktioniert nicht zufriedenstellend, die Regelung der PWM ist nicht fertig implementiert. Die Unterbereiche der Software sind zwar einzeln ansteuerbar, funktionieren jedoch im Zusammenspiel noch nicht 100\%. \\
\\
Als \textbf{weiterführende Arbeit} könnte die Motoransteuerung verbessert und zur Funktionstüchtigkeit gebracht werden. Dazu könnte unser Konzept als Grundlage genutzt werden. Weiter könnte für den Akkulader die Software funktionstüchtig gemacht werden (Implementierungsfehler ausmerzen, Regler fertig implementieren). Anschliessend müsste der gesamte Akkulader einem Test unterzogen werden.
Der Magic Glove könnte kompakter gestaltet werden, so dass der Tragkomfort gesteigert werden kann. Die Messwerte könnten besser gefiltert werden. Zur feineren Steuerung bräuchte es einige Optimierungen in der Regelung.\\
\\
Insgesamt startete das Projekt sehr dynamisch und wurde mit viel Einsatz durchgezogen. Die Probleme in der Motoransteuerung sind enttäuschend, insbesondere da keine klare Ursache gefunden werden konnte und das Konzept nicht der Fehler zu sein scheint. Dadurch kann das Skateboard nicht als Ganzes getestet und ausprobiert werden, was doch das grosse Ziel war, auf das hingearbeitet wurde. Nichtsdestotrotz wurden viele Unterziele zufriedenstellend Erreicht.