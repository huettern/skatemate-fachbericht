\chapter{Validierung} \label{Validierung}

%%%%%%%%%%%%%%%%%%%%%%%%%%%%%%%%%%%%%%%%%%%%%%%%%%%%%%%%%%%%%%%%
% Überblick
%%%%%%%%%%%%%%%%%%%%%%%%%%%%%%%%%%%%%%%%%%%%%%%%%%%%%%%%%%%%%%%%
\section{Überblick} \label{ValidUeberblick}
Ziel der Validierungsphase ist das Testen auf dem fertigen Board, anhand von Testpersonen. Der Weg bis dahin muss aber ebenfalls kontrolliert und geprüft sein. Zur Sicherstellung der Funktionalität werden die Komponenten einzeln als auch gesamthaft getestet. 
Jeder Hardware-Bestandteil – Brett, Steuerung, Stromversorgung und Motoransteuerung - hat also sein eigenes Testkonzept.

%%%%%%%%%%%%%%%%%%%%%%%%%%%%%%%%%%%%%%%%%%%%%%%%%%%%%%%%%%%%%%%%
% Brett
%%%%%%%%%%%%%%%%%%%%%%%%%%%%%%%%%%%%%%%%%%%%%%%%%%%%%%%%%%%%%%%%
\section{Brett} \label{ValidBrett}
Das Brett selber, die gepressten Birkenholzplatten, dessen Bearbeitung und die darauf befindliche Stromleiter und Gehäuse wurden einem Stabilitätstest unterzogen, wo geprüft wurde, ob die Komponenten ein Ausreizen der Flexibilität des Brettes vertragen. Die Verbindungen wurden durchgemessen. Es wurde keine Widerstandserhöhung festgestellt. Das Brett selber ist auch nicht zerbrochen.

%%%%%%%%%%%%%%%%%%%%%%%%%%%%%%%%%%%%%%%%%%%%%%%%%%%%%%%%%%%%%%%%
% Magic Glove
%%%%%%%%%%%%%%%%%%%%%%%%%%%%%%%%%%%%%%%%%%%%%%%%%%%%%%%%%%%%%%%%
\section{Steuerung - Magic Glove} \label{ValidSteuerMagicGlove}
Für die Validierung des Magic Glove wurde eine Hand 3d-gedruckt und ein seperater Empfänger mit einem Arduino gebaut, welcher die empfangenen Daten an einen Computer sendet. Die Hand wird nun von Minima zu Maxima gebeugt und die gemessenen Werte aufgezeichnet. Somit kann garantiert werden, dass der maximale Bewegungsradius aufgelöst wird und die Daten korrekt gesendet werden.
%%%%%%%%%%%%%%%%%%%%%%%%%%%%%%%%%%%%%%%%%%%%%%%%%%%%%%%%%%%%%%%%
% Stromversorgung
%%%%%%%%%%%%%%%%%%%%%%%%%%%%%%%%%%%%%%%%%%%%%%%%%%%%%%%%%%%%%%%%
\section{Stromversorgung} \label{ValidStromversorgung}

%%%%%%%%%%%%%%%%%%%%%%%%%%%%%%%%%%%%%%%%%%%%%%%%%%%%%%%%%%%%%%%%
% MC
%%%%%%%%%%%%%%%%%%%%%%%%%%%%%%%%%%%%%%%%%%%%%%%%%%%%%%%%%%%%%%%%
\section{Motoransteuerung} \label{ValidMotoransteuerung}
Um reproduzierbare Ergebnisse zu erzielen, wird die Motoransteuerung auf einem Prüfstand getestet. Der Motor wird ohne Last befestigt und die Schaltung von einem Netzteil gespiesen. Weiter wird die Motoransteuerung in einzelnen Blöcken validiert. Dabei wird unterschieden in Hardware und Software.\\
\\
\textbf{Hardware}
\begin{itemize}
	\item Funktion der FETs
	\item Spannungsmessung mit Spannungsteiler
\end{itemize}
\textbf{Software}
\begin{itemize}
	\item Einlesen der Spannungen und Ströme
	\item SVPWM Raumvektormodulation
	\item Positions- und Geschwindigkeitsbeobachter
	\item D und Q Stromregler
\end{itemize}

\subsection*{Hardware}
\subsubsection*{FETs}
Die FETs werden der Reihe nach von der Software eingeschalten. Zum Test der FETs wird am Ausgang ein Widerstand auf Masse geschaltet. So kann die Spannung am Ausgang gemessen und aufgezeichnet werden. Zusätzlich wird die Gatespannung gemessen.

\begin{center}
	\begin{tabular}{|c|c|}
		\hline 
		Versorgungsspannung & $15V$ \\ \hline
		Lastwiderstand & $1k\Omega$ \\ \hline
		Gatestrom Treiber & $1.7A$ \\ \hline
	\end{tabular} 
	\captionof{table}{Messbedingungen FETs}
	\label{tab:fetmessbed}
\end{center}

Zur Gunsten der Übersichtlichkeit wird nur die Messung eines FETs dargestellt.

\begin{figure} [H]
	\centering
	\includegraphics[width=0.5\linewidth]{images/valmcfet.jpg}
	\caption{Einschalten High-Side FET}
	\label{fig:hsfet}
\end{figure}
\begin{center}
	\begin{tabular}{|l|l|l|}
		\hline 
		Kanal 1 & High FET A Gate & Mit 10x Abschwächung\\ \hline
		Kanal 2 & Spannung Phase A & {}\\ \hline
	\end{tabular}
\end{center}

Der Kanal 1 zeigt die Steuerspannung am Gate des FETs. Sie wird vom Treiber IC erzeugt und beträgt im eingeschalteten Zustand 24V, das sind rund 7V Gate-Source Spannung. Dies versichert ein schneller Durchschalten des FETs. Erkennbar ist dies an der schnellen Flank am Ausgang der Halbbrücke, zu sehen auf Kanal 4.

\subsubsection*{Spannungsmessung}
Ist der Microkontroller im Resetzustand, sind alle FETs ausgeschaltet. So kann eine Spannung an die Ausgänge der H-Brücke gegeben werden und am Ausgang der Spannungsteiler die Spannung gemessen werden. Diese darf bei einem bestimmten Eingangsspannungsbereich die Microkontrollerspeisung nicht überschreiten.

\begin{center}
	\begin{tabular}{|c|c|}
		\hline 
		Versorgungsspannungsbereich & $0$ bis $20V$ \\ \hline
		Microkontrollerspeisung & $3.3V$ \\ \hline
		Spannungsteiler Faktor & $0.0534$ \\ \hline
	\end{tabular} 
	\captionof{table}{Messbedingungen Spannungsmessung}
	\label{tab:vmessbed}
\end{center}

\begin{center}
	\begin{tabular}{|l|l|l|}
		\hline 
		Spannung an Phase & Spannung gemessen am Microkontroller & Faktor\\ \hline
		5V & 0.268V & 0.0536\\ \hline
		10V & 0.536V & 0.0536\\ \hline
		15V & 0.804V & 0.0536\\ \hline
		20V & 1.072V & 0.0536\\ \hline
	\end{tabular} 
	\captionof{table}{Spannungsmessung Spannungsteiler}
	\label{tab:spannteiler}
\end{center}

Die Spannungen gemessen nach den Spannungsteiler entsprechen exakt dem angelegten Wert mal dem Teilungsfaktor. Dies ist erstaunlich exakt und völlig innerhalb der Widerstandstoleranzen.

\subsection*{Software}
\subsubsection*{Einlesen der Spannungen}
Über die Shell des Microkontrollers werden alle gemessenen Spannungen periodisch ausgegeben. Um die Spannungen zu messen werden die FETs ausgeschalten und eine Spannung an den Ausgängen angelegt. Um die Strommessung zu validieren werden die Low-Side FETs eingeschalten und ein Strom an den Ausgängen eingespiesen. So kann der gesamte Signalpfad der Messungen validiert werden.

\begin{center}
	\begin{tabular}{|c|c|}
		\hline 
		Testspannung & $0$ bis $20V$ \\ \hline
		Teststrom & $0$ bis $2A$ \\ \hline
	\end{tabular} 
	\captionof{table}{Messbedingungen Spannungsmessung Software}
	\label{tab:swvmessbed}
\end{center}

\begin{center}
	\begin{tabular}{|l|l|}
		\hline 
		Spannung an Phase & Spannung gemessen vom Microkontroller \\ \hline
		5V & 3.7V\\ \hline
		10V & 8.5V\\ \hline
		15V & 13.3V\\ \hline
		20V & 18.0\\ \hline
	\end{tabular} 
	\captionof{table}{Spannungsmessung Software}
	\label{tab:spannsw}
\end{center}

Die Spannungswerte weichen stark von den Sollwerten ab. Dies ist aber nicht weiter tragisch, da viel mehr der Spannungsunterschied von relevanz ist. Zudem wird bim FOC Verfahren nur die Versorgungsspannung gemessen.

\begin{center}
	\begin{tabular}{|l|l|}
		\hline 
		Strom & Gemessen vom Microkontroller \\ \hline
		0.5A & 0.587A\\ \hline
		1A & 1.007A\\ \hline
		1.5A & 1.511A\\ \hline
		2A & 1.930A\\ \hline
	\end{tabular} 
	\captionof{table}{Strommessung Software}
	\label{tab:stromsw}
\end{center}

Auch bei den Stromwerten ist der relative Unterschied wichtiger als der Absolutwert.

\subsubsection*{SVPWM Raumvektormodulation}
Für diese Validierung wird ein Sinusförmige Spannung mit konstanter Frequenz und Amplitude berechnet und der SVM routine übergeben. Die Ein- und Ausgabedaten werden für eine Zeitdauer aufgezeichnet und dann an den Computer übertragen wo sie dargestellt werden.

\begin{figure} [H]
	\centering
	\includegraphics[width=0.8\linewidth]{images/valmcsvm.png}
	\caption{Validierung SVPWM Raumvektormodulation}
	\label{fig:svpwm}
\end{figure}

Im unteren Plot der Abbildung \ref{fig:svpwm} ist sehr gut der dreiphasige Sinus mit dritter harmonischer Welle zu sehen. Diese Werte entsprechen den Dutycyclen der MOSFETs.

\subsubsection*{Positions- und Geschwindigkeitsbeobachter} \label{val:obs}
Nun wird der Motor zwangskommutiert. Das heist, dass wieder eine Sinusförmige Spannung mit konstanter Frequenz und Amplitude berechnet wird und auf die H-Brücke geführt wird. Der Motor dreht nun mit einer konstanter Drehzahl. Die Ausgangswerte der Positions- und Geschwindigkeitsbeobachter werden erneut aufgezeichnet und mit dem Computer ausgewertet.

\begin{center}
	\begin{tabular}{|c|c|}
		\hline 
		$v_{d,set}$ & $0.0$ \\ \hline
		$v_{q,set}$ & $0.07$ \\ \hline
		$f_{set}$ & $35.0Hz$ \\ \hline
		Versorgungsspannung & $15V$ \\ \hline
	\end{tabular} 
	\captionof{table}{Messbedingungen Spannungsmessung Software}
	\label{tab:obsmessbed}
\end{center}

\begin{figure} [H]
	\centering
	\includegraphics[width=0.8\linewidth]{images/valmcobserver.png}
	\caption{Validierung Positions- und Geschwindigkeitsbeobachter}
	\label{fig:observer}
\end{figure}

Abbildung \ref{fig:observer} zeigt in den ersten beiden Plots die Eingabewerte des Positionsbeobachters. Die gemessenen Ströme sind ungefiltert dargestellt und nur von kleinem Rauschen behaftet. Die Positionsschätzung theta hat ebenfalls nur einen kleinen Rippel und kann gut für die benötigten Transformationen verwendet werden.
Was in diesem Versuch nicht validier werden kann, ist der Phasenversatz zwischen wahrem und geschätztem Winkel. Dazu wären unter Anderem ein Drehgeber am Motor und eine Datenauswertung nötig, welche zeitkritisch die berechneten und gemessenen Grössen aufzeichnen kann.

Die Schätzung der Drehzahl schwingt um den Mittelwert von -300rpm was genau dem eingestellten Wert entspricht:

\begin{equation}
	\omega_m [rpm] = \frac{60 \cdot \omega_e}{p} = \frac{60 \cdot 35Hz}{7} = 300rpm
\end{equation}

Die Welligkeit wird vor dem Verwendung für den Stromregler tiefpassgefiltert.

\subsubsection*{D und Q Stromregler}
Der letzte Validierungsschritt für die Motorsteuerung auf dem Prüfstand ist die Überprüfung der D und Q Stromregler. Wie bei Validierungsschritt \ref{val:obs} werden die Daten der Regler auf dem Microkontroller zwischengespeichert und anschliessend auf dem Computer dargestellt. Bei dieser Validierung wird der Motor im Closed-Loop Modus betrieben. Es wird softwaremässig ein Sollwertsprung ausgeführt.

\begin{center}
	\begin{tabular}{|c|c|}
		\hline 
		$i_{d,set}$ & $0$ \\ \hline
		$i_{q,set}$ & 3 auf 5 \\ \hline
		Versorgungsspannung & $21V$ \\ \hline
	\end{tabular} 
	\captionof{table}{Messbedingungen D und Q Stromregler}
	\label{tab:regmessbed}
\end{center}

\begin{figure} [H]
	\centering
	\includegraphics[width=0.8\linewidth]{images/valmccontrollers.png}
	\caption{Validierung D und Q Stromregler}
	\label{fig:reg}
\end{figure}

Abbildung \ref{fig:reg} zeigt die Sprungantwort der beiden Stromregler. Beide sind nicht perfekt und könnten noch optimiert werden. Da der Motor jedoch bei hohen Drehzahlen noch nicht dreht, wurde auf eine ausführliche Optimierung verzichtet. Was jedoch bestätigt werden kann ist, dass der Regler eine schnelle Sprungantwort hat und keine bleibende Regelabweichung dank I-Anteil vorhanden ist.


\subsubsection*{Fazit}
\todo{Anderer Titel?}
Die meisten Komponenten der Motorsteuerung funktionieren. Der Motor dreht jedoch nur bei langsamen Drehzahlen (kleiner 1000rpm) und mit wenig Drehmoment. Der Fehler liegt in der Implementierung des Stromreglers. Vermutlich ist es nur noch eine Sache der richtigen Parameter. Da das Umsetzen sehr zeitaufwändig ist, wurde darauf verzichtet.


%%%%%%%%%%%%%%%%%%%%%%%%%%%%%%%%%%%%%%%%%%%%%%%%%%%%%%%%%%%%%%%%
% Gesamtvalidierung
%%%%%%%%%%%%%%%%%%%%%%%%%%%%%%%%%%%%%%%%%%%%%%%%%%%%%%%%%%%%%%%%
\section{Gesamtvalidierung}
\label{ValidGesamtv}
Nach eingehendem Testen der einzelnen Komponenten wird das Gesamtprodukt getestet. Dabei wird primär die Lauffähigkeit des Longboards und die Zusammenarbeit der einzelnen Komponenten untereinander untersucht. Insbesondere werden die im Pflichtenheft festgelegten Kriterien überprüft. \\
\textbf{Steuerung:} Schaltet der Motor aus, wenn der Benutzer nicht mehr auf dem Longboard steht? Wie gross ist die maximale Distanz, über die die Funkübertragung noch funktioniert? \\
\textbf{Stromversorgung:} Funktioniert das aktive Balancing während dem Ladevorgang? Dazu werden die Zellspannungen während eines Ladevorganges überprüft und miteinander verglichen. Existiert der Überstromschutz?\\ \todo{gehört zur Stromvers-Valid! > gesamtvalid.frage??}
\textbf{Motoransteuerung:} Ist das Fahren des Longboards ohne aktivierter Antrieb möglich? \\
\\
Da das Longboard nicht fahrtüchtig ist, konnte die Gesamtvalidierung nicht gemacht werden, deshalb sind keine Ergebnisse vorhanden.

\section{Alltagstauglichkeit}
\label{ValidAlltag}
Wie bereits angetönt, soll das fertige Board anhand von Testpersonen unterschiedlicher Skate-Erfahrung getestet werden. Die sanfte Anfahrmöglichkeit, die Manövrierfähigkeit und das allgemeine Wohlbefinden des Skaters sind die entscheidenden Kriterien. 
Testpersonen wurden zufällig ausgewählt und die Bewertungen erfolgten nach subjektivem Einschätzen.
\todo{unterschiedliche Zeiten (soll getestet werden - wurden ausgewählt / bereits getestet)}

