\chapter{Validierung} 
\label{Validierung}
\section{Überblick}
\label{ValidUeberblick}
Ziel der Validierungsphase ist das Testen auf dem fertigen Board, anhand von Testpersonen. Der Weg bis dahin muss aber ebenfalls kontrolliert und geprüft sein. Zur Sicherstellung der Funktionalität werden die Komponenten einzeln als auch gesamthaft getestet. 
Jeder Hardware-Bestandteil – Brett, Steuerung, Stromversorgung und Motoransteuerung - hat also sein eigenes Testkonzept.
\section{Brett}
\label{ValidBrett}
Das Brett selber, die gepressten Birkenholzplatten, dessen Bearbeitung und die darauf befindliche Stromleiter und Gehäuse wurden einem Stabilitätstest unterzogen, wo geprüft wurde, ob die Komponenten ein Ausreizen der Flexibilität des Brettes vertragen. Die Verbindungen wurden durchgemessen. Es wurde keine Widerstandserhöhung festgestellt. Das Brett selber ist auch nicht zerbrochen.
\section{Steuerung - Magic Glove}
\label{ValidSteuerMagicGlove}
\section{Stromversorgung}
\label{ValidStromversorgung}
\section{Motoransteuerung}
\label{ValidMotoransteuerung}
\section{Gesamtvalidierung}
\label{ValidGesamtv}
Wie bereits angetönt, soll das fertige Board anhand von Testpersonen unterschiedlicher Skate-Erfahrung getestet werden. Die sanfte Anfahrmöglichkeit, die Manövrierfähigkeit und das allgemeine Wohlbefinden des Skaters sind die entscheidenden Kriterien. 
Testpersonen wurden zufällig ausgewählt und die Bewertungen erfolgten nach subjektivem Einschätzen.
\todo{unterschiedliche Zeiten (soll getestet werden - wurden ausgewählt / bereits getestet)}
\section{Alltagstauglichkeit}
\label{ValidAlltag}