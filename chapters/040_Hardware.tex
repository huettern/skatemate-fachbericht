\chapter{Hardware}
\label{Hardware}
\section{Überblick}

Vielleicht könnte dieses Schema hier viel gewinnbringender eingefügt werden als im Abschnitt Grobkonzept?

\begin{figure}[H]
	\centering
	\includegraphics[width=1\linewidth]{images/Grobkonzept_Blockschaltbild_detailliert}
	\caption[Detailliertes Blockschaltbild]{Detailliertes Blockschaltbild}
	\label{fig:grobkonzeptblockschaltbilddetailliert_2}
\end{figure}


\section{Steuerung - Magic-Glove}
\label{HW_MagicGlove}

\section{Stromversorgung}
\label{HW_Stromversorgung}
An die Stromversorgung werden einige Anforderungen gestellt. So muss sie für die Motorsteuerung grosse Ströme mit bis zu 50A liefern können. Dabei muss die Spannung der Zellen kontinuierlich überprüft werden um ein Tiefentladen zu verhindern. Des Weiteren wird eine Standby-Killer Schaltung implementiert und dessen Funktion erklärt.\\
Um das Board einfach laden zu können ohne jedes Mal den Akku ausbauen zu müssen, hat sich unser Team entschieden eine Akku Ladeschaltung einzubauen. Diese muss die Zellen balancen und sie vor Überspannung bzw. Überladung schützen. 
\todo{Leserführung was kommt nun}
\subsection*{Balancing}
Das Balancing, wie man schon im Kapitel \ref{TechnischeGrundlagen} erfahren hat, ist eine wichtige Funktion damit der Akku nicht zerstört wird. Das ganze Balancing sollte möglichst einfach aufgebaut sein und mit dem Mikrocontroller ansteuerbar sein. Wie man in Abbildung \ref{fig:Balancing} sieht, wurde dies mit einer P-Kanal Mosfet Schaltung ermöglicht. Der Widerstand am Drain des P-Kanal Mosfet reguliert den Strom mit welchem die einzelnen Zellen ausgeglichen werden. Damit die Schaltung mit einem Mikrocontroller gesteuert werden kann, brauch es einen N-Kanal Mosfet der das Gate des P-Kanal Mosfet auf Ground zieht. Mit dieser einfachen Schaltung ist es möglich die sechs Zellen des Akku zu balancen.

\begin{figure} [H]
	\centering
	\includegraphics[width=0.6\linewidth]{images/Balancing}
	\caption{Balancing}
	\label{fig:Balancing}
\end{figure}




\subsection*{Ladeelektronik}
Damit der Akku fachgerecht geladen wird, braucht es eine Ladeelektronik, die den LiPo-Akku nach dem CCCV-Verfahren (constant current, constant voltage) lädt. Es gab zwei Möglichkeiten, diese aufzubauen. Die eine Möglichkeit wäre ein teures IC gewesen, welches auf dem Markt verfügbar ist. Die andere Möglichkeit ist eine günstige Buck-Converter Schaltung, mit der man nach dem oberen Verfahren laden kann. Ein solcher Buck-Converter ist in diesem Produkt eingesetzt.

\begin{figure} [H]
	\centering
	\includegraphics[width=0.6\linewidth]{images/Buck_Converter}
	\caption{Buck-Converter}
	\label{fig:Buck_Converter}
\end{figure}

Das Schema des Buck-Converters in Abbildung \ref{fig:Buck_Converter} zeigt, dass dieser einfach als auch schnell aufgebaut ist. Der Widerstand ist ein Shunt-Widerstand, an dem der Strom gemessen wird. Anhand dieses Stromes wird der PWM, welcher über einen High-Side-Treiber am MOSFET anliegt, durch den Mikrocontroller geregelt. Die Taktfrequenz des Clocks liegt bei 31.3kHz und ist mit der Formel \eqref{eqn:Taktfrequenz} aus dem Datenblatt des ATmega berechnet \cite{DatenblattATMEGA}.

\begin{equation}
f_{ PWM }=\frac { { f }_{ \mu C } }{ N\cdot256 } 
\label{eqn:Taktfrequenz}
\end{equation}

Mit dieser Taktfrequenz kann man nun die Grösse der Spule definieren. Dies wird mit folgender Formel \eqref{eqn:Spule} berechnet.

\begin{equation}
L=\frac { RF\cdot\left( { V }_{ IN }-{ V }_{ OUT } \right) \cdot\frac { { V }_{ OUT } }{ { V }_{ IN } }  }{ { f }_{ s }\cdot{ I }_{ OUT } }  
\label{eqn:Spule}
\end{equation}

Berechnet wurde eine Spule von rund 120$\mu$H. Eingesetzt ist eine Coilcraft-Spule mit dem Wert von 100$\mu$H. Da die Batterie wie ein grosser geladener Kondensator wirkt, sind keine all zu grossen Ripple am Ausgang des Buck-Converter zu erwarten.

\subsection*{Selbsterhaltung}
Wenn der LiPo-Akku voll geladen ist oder die Fahrt für mehr als 20min unterbrochen wird, sollte der Mikrocontroller sich selbstständig herunterfahren und die gesamte Stromversorgung möglichst keinen Strom brauchen. Dafür wurde eine Selbsterhaltungsschaltung entwickelt (Abbildung \ref{fig:Selbsterhaltung}). Sobald beim Mikrocontroller einer von beiden Zustände (langer Stillstand oder voll geladen) eintrifft, wird der N-Kanal-MOSFET ausgeschaltet. Somit wird die Speisung auf dem gesamten Print unterbrochen. Dadurch wird die Batterie bei einer längeren Lagerung vor dem Zerstören geschützt.

\begin{figure} [H]
	\centering
	\includegraphics[width=0.6\linewidth]{images/Selbsterhaltung}
	\caption{Selbsterhaltung}
	\label{fig:Selbsterhaltung}
\end{figure}






\subsection*{Mikrocontroller}
An den Mikrocontroller werden verschiedene Anforderungen gesetzt.
In folgender Tabelle \ref{tabGPIObat} werden die General Purpose Input Output (GPIO) Anforderungen an den Mikrocontroller aufgelistet.
\begin{center}
	\begin{tabular}{|c|c|c|c|}
		\hline 
		Beschreibung & Anforderung & Input/Output & Anzahl \\ 
		\hline 
		Überwachung der Zellen & Analog & Input & 6 \\ 
		\hline 
		Überwachen des Ladestroms &	Analog & Input & 1 \\ 
		\hline 
		Überwachen der Eingangsspannung & Analog & Input & 1 \\ 
		\hline 
		&  &  &  \\ 
		\hline 
		Entladen der Zellen (Balanceing) & Digital  & Output & 6 \\ 
		\hline 
		Anzeige des Ladezustands (LED) & Digital & Output & 3 \\ 
		\hline 
		Schalten des Ausgang Stroms zum Motor & Digital & Output & 1 \\ 
		\hline 
		Selbsthaltung des Mikrocontrollers & Digital & Output & 1 \\ 
		\hline 
		&  &  &  \\ 
		\hline 
		Regelung des Ladestroms & PWM & Output & 1 \\ 
		\hline 
		Kommunikation SPI Schnittstelle MISO/MOSI/SCK & Serial & In/Output & 3 \\ 
		\hline 
	\end{tabular} 
	\captionof{table}{GPIO Anforderungen für den Mikrocontroller}
	\label{tabGPIObat}
\end{center}
Für die Regelung der Stromversorgung wird der Mikrocontroller ATmega328P-AU verwendet.
\subsection*{Zellmessung}
Die Messung der Zellen durchlief während der Entwicklungsphase mehrere Änderungen. Die Zellen sind in Serie geschaltet, um dem Motor genügend Spannung zur Verfügung zu stellen. Mit einer Zellenspannung von 4.15V pro Zelle liegt am obersten Punkt (Anode der Zelle 6) eine Spannung von 24,9V. Da die Spannung an den Eingängen des Mikrocontrollers nicht seine Speisespannung übersteigen sollte, mussten diese Spannungen verringert werden. In unserem Fall beträgt die maximale Spannung am Eingang 5V. Die erste Idee war, wie in Abb.\ref{fig:zellmessung} links oben zu sehen, mit einfachen Spannungsteiler die Spannungen so zu skalieren, dass die Maximalspannung nicht überschritten wird. 
\begin{figure} [H]
	\centering
	\includegraphics[width=1\linewidth]{images/Zellmessung}
	\caption{Messung der Zellen - verschiedene Arten}
	\label{fig:zellmessung}
\end{figure}
\todo{besserer Titel für Abb.}
Der Nachteil dieser Schaltung ist, dass die Zellen durch die Spannungsteiler kontinuierlich entladen werden. Somit können bei einem längeren Nichtgebrauch die Akkuzellen entladen oder sogar tiefentladen sein. \\
Um dies zu verhindern wurde der zweite Entwurf auf der rechten Seite der Abb.\ref{fig:zellmessung} entwickelt. Dieser unterbricht mit den Transistoren den Entladestromkreis über den Widerständen. Die Transistoren werden übersteuert, um sie als Schalter zu verwenden und den durch die UCE \todo{hä? bzw. Formatierung notwendig? U$_{CE}$ --> U\_collector\_emitter aaahhhaaa} Spannung entstehende Fehler zu verkleinern. Dieser tritt je nach Sättigungsspannung der Transistoren jedoch immer noch auf und ist durch die unterschiedlichen Bauarten nur schwer zu korrigieren. Der zweite grosse Nachteil und gleichzeitig Killerkriterium für diese Schaltung war, dass durch das Unterbrechen des Spannungsteiler Stromkreises an den ADC Ausgängen wieder die Zellenspannung anliegt. Dadurch würden sich die Zellen über die Ableitdiode am Eingang des Mikrocontrollers entladen und diese \todo{diese = die Ableitdiode?} voraussichtlich zerstören. \\
Dies könnte verhindert werden, indem der Stromkreis oberhalb der ADC Eingängen unterbrochen wird. Dies ist in der linken unteren Schaltung der Abb.\ref{fig:zellmessung} gezeichnet. Dadurch wären die Eingänge vor Überspannung geschützt. Aber auch diese Schaltung bringt diverse Nachteile. So fliesst beim geschlossenen Zustand der Basis-Emitter Strom durch den Spannungsteiler. Dies verfälscht die Messresultate. Zusätzlich muss zwischen Basis und Emitter eine Schaltspannung von mindestens 0.7V sein. Dazu müsste die Schaltspannung 0.7V über der Zellspannung liegen, was nur mit grossem Aufwand erreichbar wäre. 
In einem Teamentscheid wurde die erste Methode gewählt, da sie die genausten Messungen liefert und die Entladung der Zellen bei einer durchschnittlichen Benutzung des Boards nie \todo{ev.:nicht ?} zu tragen kommt.

\section{Motoransteuerung}
\label{HW_Motoransteuerung}
\subsection*{Treiber IC und Speisung}
\subsection*{Endstufe (H-Brücke)}
\subsection*{Messschaltung}
\subsection*{Mikrocontroller}

